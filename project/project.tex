\documentclass{article}

\usepackage{fancyhdr}
\usepackage{extramarks}
\usepackage{amsmath}
\usepackage{amsthm}
\usepackage{amsfonts}
\usepackage{tikz}
\usepackage{listings}
\usepackage[plain]{algorithm}
\usepackage{algpseudocode}
\usepackage{enumerate}
\usepackage{tikz}
\usepackage{xifthen}
\usepackage{xparse}
\usepackage{amsmath, amssymb}
\usepackage{lipsum}

\usetikzlibrary{automata,positioning}

%
% Basic Document Settings
%  

\topmargin=-0.45in
\evensidemargin=0in
\oddsidemargin=0in
\textwidth=6.5in
\textheight=9.0in
\headsep=0.25in

\linespread{1.1}

\pagestyle{fancy}
\lhead{\hmwkAuthorName}
\chead{\hmwkClass: \hmwkTitle}
\rhead{\firstxmark}
\lfoot{\lastxmark}
\cfoot{\thepage}

\renewcommand\headrulewidth{0.4pt}
\renewcommand\footrulewidth{0.4pt}

\setlength\parindent{0pt}

%
% Create Problem Sections
%

\newcommand{\enterProblemHeader}[1]{
    \nobreak\extramarks{}{Problem \arabic{#1} continued on next page\ldots}\nobreak{}
    \nobreak\extramarks{Problem \arabic{#1} (continued)}{Problem \arabic{#1} continued on next page\ldots}\nobreak{}
}

\newcommand{\exitProblemHeader}[1]{
    \nobreak\extramarks{Problem \arabic{#1} (continued)}{Problem \arabic{#1} continued on next page\ldots}\nobreak{}
    \stepcounter{#1}
    \nobreak\extramarks{Problem \arabic{#1}}{}\nobreak{}
}

\newcommand*\circled[1]{\tikz[baseline=(char.base)]{
		\node[shape=circle,draw,inner sep=2pt] (char) {#1};}}


\setcounter{secnumdepth}{0}
\newcounter{partCounter}
\newcounter{homeworkProblemCounter}
\setcounter{homeworkProblemCounter}{1}
\nobreak\extramarks{Problem \arabic{homeworkProblemCounter}}{}\nobreak{}

%
% Homework Problem Environment
%
% This environment takes an optional argument. When given, it will adjust the
% problem counter. This is useful for when the problems given for your
% assignment aren't sequential. See the last 3 problems of this template for an
% example.
%

\NewDocumentEnvironment{homeworkProblem}{s m}{
    \IfBooleanT{#1}{\newpage}
    \section{Problem \arabic{homeworkProblemCounter} {\small (#2)}}
    \setcounter{partCounter}{1}
    \enterProblemHeader{homeworkProblemCounter}

}{
    \exitProblemHeader{homeworkProblemCounter}
}

%
% Homework Details
%   - Title
%   - Due date
%   - Class
%   - Instructor
%   - Class number
%   - Name
%   - Student ID

\newcommand{\hmwkTitle}{Final Project}
\newcommand{\hmwkDueDate}{December 31, 2022}
\newcommand{\hmwkClass}{Probability and Mathematical Statistics}
\newcommand{\hmwkClassInstructor}{Professor Ziyu Shao}

%\newcommand{\hmwkClassID}{}

\newcommand{\hmwkAuthorName}{Ren Hui, Wanchen Su}
\newcommand{\hmwkAuthorID}{xxxxxxx, 2021533067}

%
% Title Page
%

\title{
    \vspace{2in}
    \textmd{\textbf{\hmwkClass:\\  \hmwkTitle}}\\
    \normalsize\vspace{0.1in}\small{Due\ on\ \hmwkDueDate\ at 11:59am}\\
   %\vspace{2in}\Huge{\hmwkClassID}\\   
   \vspace{2in}
}

\author{
	Name: \textbf{\hmwkAuthorName} \\
	Student ID: \hmwkAuthorID}
\date{}


\renewcommand{\part}[1]{\textbf{\large Part (\alph{partCounter})}\stepcounter{partCounter}\\}

%
% Various Helper Commands
%

% Useful for algorithms
\newcommand{\alg}[1]{\textsc{\bfseries \footnotesize #1}}
% For derivatives
\newcommand{\deriv}[1]{\frac{\mathrm{d}}{\mathrm{d}x} (#1)}
% For partial derivatives
\newcommand{\pderiv}[2]{\frac{\partial}{\partial #1} (#2)}
% Integral dx
\newcommand{\dx}{\mathrm{d}x}
% Alias for the Solution section header
\newcommand{\solution}{\textbf{\Large Solution}}
% Probability commands: Expectation, Variance, Covariance, Bias
\newcommand{\E}{\mathrm{E}}
\newcommand{\Var}{\mathrm{Var}}
\newcommand{\Cov}{\mathrm{Cov}}
\newcommand{\Bias}{\mathrm{Bias}}

\begin{document}

\maketitle
\pagebreak

% Problem 1
\begin{homeworkProblem}{Part 1: classical Bandit Algorithms}
    \solution
    \begin{enumerate}
        \item[1,2.]
        See jupyter notebook.
        \item[3.]
        Accorrding to python simulation, the results are repectively:\\
        results for epsilon-greedy Algorithm:\\
        $[0.69956534\ 0.50167567\ 0.39846044]$ with parameter: 0.1\\
        $[0.70064482\ 0.49688743\ 0.39858856]$ with parameter: 0.5\\
        $[0.70009067\ 0.49985083\ 0.39811366]$ with parameter: 0.9\\
        results for UCB Algorithm:\\
        $[0.70088294\ 0.49274932\ 0.38448981]$ with parameter: 1\\
        $[0.7006747\ 0.49900798\ 0.39886189]$ with parameter: 5\\
        $[0.70135645\ 0.49877856\ 0.40023188]$ with parameter: 10\\
        results for TS Algorithm:\\
        $[0.6995964\ 0.45887541\ 0.37554019]$ with parameter: [[1, 1], [1, 1], [1, 1]]\\
        $[0.68296885\ 0.4001996\ 0.362563]$ with parameter: [[601, 401], [401, 601], [2, 3]]
        \item[4.]
        In the $\epsilon$-greedy algorithm, $\epsilon$ decides the probability of choosing the max estimate of all the arms or choosing a random arm.
        Which means the larger $\epsilon$ is, the more evenly spread are the tests.\\
        In the UCB algorithm, the parameter c ...\\
        If the initial value of $\alpha_j,\beta_j$ we passed in is too small,
        then the result of the first few tests may influence the final result greatly,
        and  only a few dozen tests are on the second and third arm. If this happens, the result will be frightfully small.
        And after 200 independent trials of $N = 5000$, the final estimation still has a great gap with the oracle value.
        On the other hand, if the initial value of $\alpha_j,\beta_j$ we passed in is relativly large, this value of a and b may influence the final result greatly.
        As in the trial with parameter [[601, 401], [401, 601], [2, 3]], we can see that the final estimation of arm two is very close to $\frac{401}{401+601}$.
        More impacts of $\epsilon$, c and $\alpha_j,\beta_j$ concerning exploration-exploitation will be disccussed in 5.
        \item[5.]
        In the case of all algorithms, more exploitation means that the total reward we get is larger, while more exploration means
        the estimated value of the reward of each arm is more accurate, but it also means that the total reward during our tests is less.\\
        For the $\epsilon$-greedy algorithm, the larger $\epsilon$ is, there is more chance of exploration than exploitation.
        So while $\epsilon$ grows, the gaps between the algorithm output and the oracle value decreases, but the sum of the reward of all trials is smaller.\\
        As for the UCB Algorithm, the exploration-exploitation trade-off depends on the value of c.
        Similar to the $\epsilon$-greedy algorithm, the larger c is, there is more exploration and less exploitation.
        As can be deduced from the data in the python simulation, for larger c, the estimated value for the third arm is more accurate.
        However, the estimated value for arms 1 and 2 becomes less accurate. After diccussion, we believe this is because the larger c is, 
        the more average the tests are. Consiquently, there are more tests on the third arm (it has the smallest oracle value), 
        but there are less test on the first and second arm.\\
        The TS Algorithm is rather different from the previous ones. Its exploration is very limited. 
        The arm selected is always the arm with the greates current reward estimation. So arms with small oracle values may end up being pulled only a few times.
        Which makes the estimate value of arms 2 and 3 rather inaccurate. But the overall reward gained within the N trials should be the largest of all the algorithms.
        \item[6.]

    \end{enumerate}
\end{homeworkProblem}
\pagebreak




\end{document}